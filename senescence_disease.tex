%%%%%%%%%%%%%%%%%%%%%%%%%%%%%%%%%%%%%%%%%%%%%%%%%%%%%%%%%%%%%%%%%%%%%%%%%%%%%%%%%%%%%%%%%%%%%%%%%%%%%%%%%%%%%%%%%%%%%%%%%%%%%%%%%%%%%%%%%%%%%%%%%%%%%%%%%%%
% This is just an example/guide for you to refer to when submitting manuscripts to Frontiers, it is not mandatory to use Frontiers .cls files nor frontiers.tex  %
% This will only generate the Manuscript, the final article will be typeset by Frontiers after acceptance.                                                 %
%                                                                                                                                                         %
% When submitting your files, remember to upload this *tex file, the pdf generated with it, the *bib file (if bibliography is not within the *tex) and all the figures.
%%%%%%%%%%%%%%%%%%%%%%%%%%%%%%%%%%%%%%%%%%%%%%%%%%%%%%%%%%%%%%%%%%%%%%%%%%%%%%%%%%%%%%%%%%%%%%%%%%%%%%%%%%%%%%%%%%%%%%%%%%%%%%%%%%%%%%%%%%%%%%%%%%%%%%%%%%%

%%% Version 3.1 Generated 2015/22/05 %%%
%%% You will need to have the following packages installed: datetime, fmtcount, etoolbox, fcprefix, which are normally inlcuded in WinEdt. %%%
%%% In http://www.ctan.org/ you can find the packages and how to install them, if necessary. %%%

\documentclass{frontiersSCNS} % for Science, Engineering and Humanities and Social Sciences articles
%\documentclass{frontiersHLTH} % for Health articles
%\documentclass{frontiersFPHY} % for Physics and Applied Mathematics and Statistics articles

%\setcitestyle{square}
\usepackage{url,hyperref,lineno,microtype}
\usepackage[onehalfspacing]{setspace}
\usepackage{subcaption}

\linenumbers


% Leave a blank line between paragraphs instead of using \\


\def\keyFont{\fontsize{8}{11}\helveticabold }
\def\firstAuthorLast{Camargo {et~al.}} %use et al only if is more than 1 author
\def\Authors{Anyela V Camargo\,$^{1,*}$, Jan T Kim\,$^{2}$ and Alison Bentley\,$^{2}$}
% Affiliations should be keyed to the author's name with superscript numbers and be listed as follows: Laboratory, Institute, Department, Organization, City, State abbreviation (USA, Canada, Australia), and Country (without detailed address information such as city zip codes or street names).
% If one of the authors has a change of address, list the new address below the correspondence details using a superscript symbol and use the same symbol to indicate the author in the author list.
\def\Address{$^{1}$National Plant Phenomics Centre, IBERS, Aberyswyth University, Gogerddan, Aberystwyth, SY23 3EB, UK\\
$^{2}$Laboratory X, Institute X, Department X, Organization X, City X , State XX (only USA, Canada and Australia), Country X  }
% The Corresponding Author should be marked with an asterisk
% Provide the exact contact address (this time including street name and city zip code) and email of the corresponding author
\def\corrAuthor{Anyela V Camargo}
\def\corrAddress{National Plant Phenomics Centre, IBERS, Aberyswyth University, Gogerddan, Aberystwyth, SY23 3EB, UK}
\def\corrEmail{avc1@aber.ac.uk}




\begin{document}
\onecolumn
\firstpage{1}

\title[Running Title]{Article Title} 

\author[\firstAuthorLast ]{\Authors} %This field will be automatically populated
\address{} %This field will be automatically populated
\correspondance{} %This field will be automatically populated

\extraAuth{}% If there are more than 1 corresponding author, comment this line and uncomment the next one.
%\extraAuth{corresponding Author2 \\ Laboratory X2, Institute X2, Department X2, Organization X2, Street X2, City X2 , State XX2 (only USA, Canada and Australia), Zip Code2, X2 Country X2, email2@uni2.edu}


\maketitle



\begin{abstract}

\tiny
 \keyFont{ \section{Keywords:} wheat senescence diseases datascience phenomics } %All article types: you may provide up to 8 keywords; at least 5 are mandatory.
\end{abstract}

\section{Introduction}

Senescence is the last developmental stage of plant cells, tissues, organs, and, in the case of monocarpic species, the entire plant. Once senescence has been initiated, it typically leads to a massive remobilization of phloem-mobile nutrients from the senescing plant parts to developing sinks, such as seeds and grains of monocarpic crops. In this context, nitrogen holds a special position. It is the quantitatively most important plant mineral nutrient, and nitrogen-containing macromolecules  have to be hydrolysed and converted to glutamate, glutamine and (to a lesser extent) other amino acids prior to phloem loading and transport to developing sinks \cite{Distelfeld27012014}.

%\begin{methods}
\section{Material \& Methods}

\subsection{Plant Material}
The eight-parent MAGIC population described by \cite{Mackay01092014} was used for all phenotypic screening. Briefly, the complete population consists of more than >1000 lines generated from three rounds of inter-crossing between eight elite United Kingdom wheat varieties (Alchemy, Brompton, Claire, Hereward, Rialto, Robigus, Soissons, Xi19) followed by five rounds of selfing to remove heterozygosity. 

\subsection{Glass House Cultivation}
An experiment with duration of six weeks was conducted between mid January 2015 and mid April 2015 in The National Plant Phenomics Centre greenhouse facilities in Aberystwyth, UK. The MAGIC parents as well as 208 lines (see Supporting Information, Table S1) were grown under a well watered conditions with two replicates per genotype. The experiment was designed using randomised blocks (Fig. 1, Table S1).

Single plants were grown in ? L plastic containers with ? kg of soil (content ?). N number of seeds? seeds per pot were directly sown into the soil and after germination thinned out, leaving one plant per pot. Plants were pre-grown for two weeks in a regular greenhouse and watering was performed manually to allow optimal germination and seedling establishment. Subsequently, the pots were transferred to the “smart house” where each pot was placed onto a cart on a conveyor belt. Every day, pots were weighed and watered automatically to ?\% gravimetric water content (Fig. 2). The experiments were carried out under natural lighting with the temperature in the greenhouse kept at a range between ?°C (night) and ?°C (day).

\subsection{Plant infection}
Wheat plants were naturally infected with tan spot which is caused by the fungus Pyrenophora tritici-repentis. Pyrenophora tritici-repentis is a necrotrophic plant pathogen of fungal origin Phylum Ascomycota. Tan spot symptoms can be manifested as necrosis, chlorosis or both. The necrosis symptom comprises spots that appear initially as tan-brown flecks and expand into lens-shaped, tan lesions with yellow borders. The chlorosis symptom consists of rapidly expanding yellow areas surrounding lesions on the leaf blades. Lesions may coalesce into large blotches as they age, predisposing leaves to premature senescence.  If infected leaves are moistened, the lesions darken at the center due to formation of conidiophores and conidia of D. tritici-repentis \cite{Singh2009}. 
\subsection{Disease scoring}
Tan spot infection was scored manually using the seedling infection type (IT) score shown in Table \ref{Tab:02}. These qualitative IT scores were converted to a numerical scale for all statistical analysis.

\subsection{phenotyping}
Plants were phenotyped from the start of the experiment. Plant images were captured using a LemnaTec 3D Scanalyzer (LemnaTec, GmbH, Wuerselen, Germany). Every day, three RGB pictures (2056×2454 pixels) were taken of each wheat plant, one top view image and two side view images with a 90° horizontal rotation. After background-foreground separation was applied to separate the plant tissue area from the background, pixel numbers per plant were counted and the pixel sum of the three pictures per plant was taken to define the biomass %{projected shoot area. The shoot area measured over time was used to draw growth curves. For each growth curve, curve fitting with a 6th order polynomial was conducted to adjust for possible missing data points and absolute growth rate [dA/dt] and relative growth rate [(dA/dt)/A] were calculated. For each of the three curves the integral was determined and used as a trait in the statistical analysis}. 
Moreover, six further traits were extracted from the images; caliper length, height, color (as hue angle in the HSI color scheme) and the two parameters shoot area top view and convex hull area to calculate compactness of each plant. At the end of the experiment, barley plants were harvested and above ground biomass, tiller number (TIL), and plant height (HEI) were determined. Fresh biomass was weighed and, subsequently, oven dried to constant weight to determine dry biomass. Water use efficiency (WUE) was calculated by dividing dry biomass at the end of the experiment by the total amount of water added during the four weeks in the “smart house” [mg/g water]. Specific plant weight (SPW) was calculated from the dry weight and the maximum projected shoot area at the end of the experiment. In addition, simple stress indices (SSI) were calculated as follows: SSI = Ts/Tc, where Ts and Tc are the average trait performances of an IL under stress and control conditions, respectively. An overview of trait definitions is given in Table . To avoid bias in the scoring, two diferent people scored the plants  at two different times and scores were averaged. 

\begin{table}[!t]
\textbf{\refstepcounter{table}\label{Tab:02} Table \arabic{table}.}{ Infection type score table}\\
\processtable{}
{\begin{tabular}{l|l}
\hline
	score & description \\\midrule
	0 & No visible symptoms \\
	1  & Small, sporulating uredia surrounded by necrotic tissue \\
	2 & Small, sporulating uredia surrounded by necrotic tissue \\
	3 & Medium sized, sporulating uredia surrounded only by chlorotic tissue \\
	4 & Large, sporulating uredia surrounded by green tissue \\ \hline
\end{tabular}}{}
\end{table}


\subsection{Statistical Analysis}
Statistical analyses were performed using the R environment \cite{RManual}. 

\section{Results}
Tan spot resistance was evaluated in the greenhouse by measuring the extent of the necrotic/chlorotic response exhibited by the plant (IT).

When comparing disease scores against the onset of senescence at the Flag Leaf we found a correlation between disease severity and early onset of senescence Fig \ref{fig:01}. This result is consistent with the way necrotrophic plant pathogens act. They promote senescence in the host by manipulating respective signaling pathways. 

We also found some lines whose disease succesptibility was high (i.e. 4) but whose senescence onset was delay Fig \ref{fig:02}. 


\section{Discussion}



\section*{Disclosure/Conflict-of-Interest Statement}


The authors declare that the research was conducted in the absence of any commercial or financial relationships that could be construed as a potential conflict of interest.

\section*{Author Contributions}

Conceived and designed the study: AVC. Performed the study: AVC. Analyzed the data: AVC JTK. Contributed with seed: AB. Wrote the paper: AVC JTK. Provided comments and corrected the manuscript: ALL.

\section*{Acknowledgments}
 We are grateful to the team of “National Plant Phenomics Centre” for carrying out the experiments and to the 
\textit{Funding\textcolon} for finantially suportting the project.

\section*{Supplemental Data}


\bibliographystyle{frontiersinSCNS_ENG_HUMS} % for Science, Engineering and Humanities and Social Sciences articles, for Humanities and Social Sciences articles please include page numbers in the in-text citations
%\bibliographystyle{frontiersinHLTH&FPHY} % for Health and Physics articles
\bibliography{bioinfo}

%%% Upload the *bib file along with the *tex file and PDF on submission if the bibliography is not in the main *tex file

\section*{Figures}

%%% Use this if adding the figures directly in the mansucript, if so, please remember to also upload the files when submitting your article
%%% There is no need for adding the file termination, as long as you indicate where the file is saved. In the examples below the files (logo1.jpg and logo2.eps) are in the Frontiers LaTeX folder
%%% If using *.tif files convert them to .jpg or .png

\begin{figure}[h!]
\begin{center}
\includegraphics[width=10cm]{s_d.png}
\end{center}
 \textbf{\refstepcounter{figure}\label{fig:01} Figure \arabic{figure}.}{Disease severity vs onset of senescence}
\end{figure}


\begin{figure}
    \begin{center}
    \begin{subfigure}[b]{0.5\textwidth}
        \includegraphics[width=\textwidth]{fig2a.png}
        \caption{Early senescence, IT = 4}
       % \label{fig:tiger}
    \end{subfigure}
    ~ %add desired spacing between images, e. g. ~, \quad, \qquad, \hfill etc. 
    %(or a blank line to force the subfigure onto a new line)
    \begin{subfigure}[b]{0.5\textwidth}
        \includegraphics[width=\textwidth]{fig2b.png}
        \caption{Late senescence, IT = 4}
        %\label{fig:mouse}
    \end{subfigure}
\end{center}
    %\caption{Pictures of animals}\label{fig:animals}
\textbf{\refstepcounter{figure}\label{fig:02} Figure \arabic{figure}.}{Contrast beween two lines whose disease score was four. Red vertical line indicates the date senescence on Flag leaf was first seen}
\end{figure}


%\begin{figure}
%\begin{center}
%\includegraphics[width=10cm]{logo2}% This is an *.eps file
%\end{center}
%\textbf{\refstepcounter{figure}\label{fig:02} Figure \arabic{figure}.}{ Enter the caption for your figure here.  Repeat as  necessary for each of your figures }
%\end{figure}

%%% If you don't add the figures in the LaTeX files, please upload them when submitting the article.

%%% Frontiers will add the figures at the end of the provisional pdf automatically %%%

%%% The use of LaTeX coding to draw Diagrams/Figures/Structures should be avoided. They should be external callouts including graphics.
\end{document}
